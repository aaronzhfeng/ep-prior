% Table 3: Concept Predictability
% Function: Demonstrate interpretability - structured latents encode physiologically relevant information
% Data Source: 06_results.md concept predictability matrix
% Key Finding: z_T shows selectivity for STTC (+0.076)

\begin{table}[t]
    \centering
    \caption{Concept predictability: AUROC for predicting superclasses from individual latent components via linear probes.}
    \label{tab:concept_pred}
    \footnotesize
    \begin{tabular}{lccccc}
        \toprule
        \textbf{Class} & $z_P$ & $z_{QRS}$ & $z_T$ & $z_{HRV}$ & \textbf{All} \\
        \midrule
        NORM & .897 & .884 & .886 & .895 & \textbf{.905} \\
        MI & .774 & .773 & .770 & .781 & \textbf{.806} \\
        STTC & .882 & .887 & \underline{.883} & .899 & \textbf{.906} \\
        CD & .786 & \underline{.789} & .797 & .801 & \textbf{.811} \\
        HYP & .762 & .774 & .774 & .778 & \textbf{.791} \\
        \bottomrule
    \end{tabular}
    \vspace{0.3em}
    
    \small{\textit{Underlined values indicate expected associations per domain knowledge ($z_{QRS} \to$ CD, $z_T \to$ STTC). $z_T$ shows positive selectivity for STTC (+0.076). Individual components achieve $>$75\% of full model performance.}}
\end{table}

